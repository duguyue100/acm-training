\documentclass{article}

\usepackage[left=2cm,right=2cm,top=2cm,bottom=2cm]{geometry}
\usepackage{graphics}
\usepackage{xspace}

\newcommand{\hangout}{\textsc{Google Hangout}\xspace}
\newcommand{\host}{\textsc{PuzzleHost}\xspace}

\title{\textbf{Standard Operating Procedure for Conducting Successful} \textsc{Google Hangout}}
\author{Puzzles}

\date{Last update: \today}

\begin{document}

\maketitle

\begin{abstract}
Currently, all communication and training sessions in Puzzles are carried out through \hangout. This article presents a Standard Operating Procedure (SOP) for conducting a successful \hangout session. All procedures have to be followed by host of \hangout session.
\end{abstract}

\tableofcontents

\section{Host}

\paragraph{1.1} All public \hangout sessions are hosted on the behalf of Amy Theia Knuth (referred as Amy here and after).

\paragraph{1.2} Amy is accessible through:
\begin{verbatim}
http://google.com/+AmyKnuthTheia/
\end{verbatim}

\paragraph{1.3} Amy is reachable through:
\begin{verbatim}
amy.theia.knuth@gmail.com
\end{verbatim}

\paragraph{1.4} Only limited number of Puzzlers have access to Amy's account. They are referred as \host here and after. The list is maintained by Amy.

\paragraph{1.5} Password of Amy's account can not be changed.

\paragraph{1.6} \host can gain access to Amy's account by emailing to Amy.

\paragraph{1.7} \host are not allowed to distribute Amy's password publicly.

\paragraph{1.8} If the \host can not perform the duty of hosting a \hangout session, the \host shall find another \host to perform the duty or call off the \hangout session.

\paragraph{1.9}

\section{}

\end{document}